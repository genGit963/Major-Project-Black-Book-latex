\chapter{RELATED THEORY}
\section{Speech recognition}

Speech recognition involves capturing sound, cleaning it up, and extracting important features. These features are then compared to models to understand what's being said. We also use language rules to help figure out the meaning. Finally, we adjust and improve our system over time. With advancements in technology, speech recognition has become more accurate and is used in many everyday applications like virtual assistants and dictation software.

\section{Image processing}
 
Image processing encompasses critical tasks such as object detection and Canny edge detection. Object detection involves identifying and locating specific objects within an image or video, essential for various applications like autonomous vehicles and surveillance.

Canny edge detection is fundamental in identifying edges in images. It follows a multi-step process involving Gaussian smoothing, gradient calculation, non-maximum suppression, and edge tracking by hysteresis. This technique is widely used in object recognition, image segmentation, and feature extraction due to its accuracy in detecting edges while minimizing noise.

Both object detection and Canny edge detection are crucial in computer vision, aiding in understanding visual information and advancing artificial intelligence applications. Object detection includes steps like feature extraction, classification, and localization, while Canny edge detection highlights regions of significant intensity change in an image.

\section{Kinematics}

Kinematics in a 6 Degree of Freedom (6DOF) robotic arm pertains to the study of motion without considering the forces that cause it. This branch of robotics focuses on understanding the arm's movement in terms of position, velocity, and acceleration. In a 6DOF robotic arm, each joint provides a degree of freedom, enabling movement along six axes: three for translation (X, Y, Z) and three for rotation (roll, pitch, yaw). Kinematic analysis involves determining the arm's end-effector position and orientation concerning the base, given the joint angles. Forward kinematics deduces the end-effector pose based on joint configurations, while inverse kinematics involves finding the joint angles required to achieve a desired end-effector pose. Various kinematic models, such as the Denavit-Hartenberg (DH) method or transformation matrices, help describe the arm's geometry and kinematic behavior. These models aid in trajectory planning, path generation, and control strategies, contributing to the arm's accurate and precise motion in complex applications like manufacturing, assembly, and automation. Understanding the kinematics of a 6DOF robotic arm is pivotal in designing efficient and effective motion planning algorithms for diverse industrial and research purposes.



